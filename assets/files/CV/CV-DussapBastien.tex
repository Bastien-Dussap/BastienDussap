%% start of file `template.tex'.
%% Copyright 2006-2015 Xavier Danaux (xdanaux@gmail.com), 2020-2022 moderncv maintainers (github.com/moderncv).
%
% This work may be distributed and/or modified under the
% conditions of the LaTeX Project Public License version 1.3c,
% available at http://www.latex-project.org/lppl/.


\documentclass[11pt,a4paper,roman]{moderncv}        % possible options include font size ('10pt', '11pt' and '12pt'), paper size ('a4paper', 'letterpaper', 'a5paper', 'legalpaper', 'executivepaper' and 'landscape') and font family ('sans' and 'roman')

% moderncv themes
\moderncvstyle{classic}                             % style options are 'casual' (default), 'classic', 'banking', 'oldstyle' and 'fancy'
\moderncvcolor{purple}                               % color options 'black', 'blue' (default), 'burgundy', 'green', 'grey', 'orange', 'purple' and 'red'
%\renewcommand{\familydefault}{\sfdefault}         % to set the default font; use '\sfdefault' for the default sans serif font, '\rmdefault' for the default roman one, or any tex font name
%\nopagenumbers{}                                  % uncomment to suppress automatic page numbering for CVs longer than one page

% adjust the page margins
\usepackage[scale=0.75]{geometry}
\setlength{\footskip}{136.00005pt}                 % depending on the amount of information in the footer, you need to change this value. comment this line out and set it to the size given in the warning
%\setlength{\hintscolumnwidth}{3cm}                % if you want to change the width of the column with the dates
%\setlength{\makecvheadnamewidth}{10cm}            % for the 'classic' style, if you want to force the width allocated to your name and avoid line breaks. be careful though, the length is normally calculated to avoid any overlap with your personal info; use this at your own typographical risks...

% font loading
% for luatex and xetex, do not use inputenc and fontenc
% see https://tex.stackexchange.com/a/496643
\ifxetexorluatex
  \usepackage{fontspec}
  \usepackage{unicode-math}
  \defaultfontfeatures{Ligatures=TeX}
  \setmainfont{Latin Modern Roman}
  \setsansfont{Latin Modern Sans}
  \setmonofont{Latin Modern Mono}
  \setmathfont{Latin Modern Math} 
\else
  \usepackage[T1]{fontenc}
  \usepackage{lmodern}
\fi

% document language
\usepackage[english]{babel}  % FIXME: using spanish breaks moderncv

% personal data
\name{Bastien}{Dussap}
\title{PhD student in mathematics}                               % optional, remove / comment the line if not wanted
\born{17 June 1998}                                 % optional, remove / comment the line if not wanted
%\address{street and number}{postcode city}{country}% optional, remove / comment the line if not wanted; the "postcode city" and "country" arguments can be omitted or provided empty
\phone[mobile]{07~77~99~21~57}                   % optional, remove / comment the line if not wanted; the optional "type" of the phone can be "mobile" (default), "fixed" or "fax"
%\phone[fixed]{+2~(345)~678~901}
%\phone[fax]{+3~(456)~789~012}
\email{bastien.dussap@inria.fr} % optional, remove / comment the line if not wanted
\homepage{bastiendussap.github.io}  % optional, remove / comment the line if not wanted

% Social icons
%\social[linkedin]{bastien-dussap-4191281b8}                        % optional, remove / comment the line if not wanted
%\social[xing]{john\_doe}                           % optional, remove / comment the line if not wanted
\social[twitter]{BastienDussap}                             % optional, remove / comment the line if not wanted
\social[github]{BastienDussap}                              % optional, remove / comment the line if not wanted
\social[gitlab]{bastiendussapapb}                              % optional, remove / comment the line if not wanted
% \social[stackoverflow]{0000000/johndoe}            % optional, remove / comment the line if not wanted
% \social[bitbucket]{jdoe}                           % optional, remove / comment the line if not wanted
% \social[skype]{jdoe}                               % optional, remove / comment the line if not wanted
% \social[orcid]{0000-0000-000-000}                  % optional, remove / comment the line if not wanted
% \social[researchgate]{jdoe}                        % optional, remove / comment the line if not wanted
% \social[researcherid]{jdoe}                        % optional, remove / comment the line if not wanted
% \social[telegram]{jdoe}                            % optional, remove / comment the line if not wanted
% \social[whatsapp]{12345678901}                     % optional, remove / comment the line if not wanted
% \social[signal]{12345678901}                       % optional, remove / comment the line if not wanted
% \social[matrix]{@johndoe:matrix.org}               % optional, remove / comment the line if not wanted
% \social[googlescholar]{sdbXVlIAAAAJ}            % optional, remove / comment the line if not wanted


%\extrainfo{additional information}                 % optional, remove / comment the line if not wanted
%\photo[64pt][0.4pt]{picture}                       % optional, remove / comment the line if not wanted; '64pt' is the height the picture must be resized to, 0.4pt is the thickness of the frame around it (put it to 0pt for no frame) and 'picture' is the name of the picture file
%\quote{Il n'y a pas besoin de pele pour dériver.}   % optional, remove / comment the line if not wanted

% bibliography adjustments (only useful if you make citations in your resume, or print a list of publications using BibTeX)
%   to show numerical labels in the bibliography (default is to show no labels)
%\makeatletter\renewcommand*{\bibliographyitemlabel}{\@biblabel{\arabic{enumiv}}}\makeatother
\renewcommand*{\bibliographyitemlabel}{[\arabic{enumiv}]}
%   to redefine the bibliography heading string ("Publications")
%\renewcommand{\refname}{Articles}

% bibliography with mutiple entries
%\usepackage{multibib}
%\newcites{book,misc}{{Books},{Others}}
%----------------------------------------------------------------------------------
%            content
%----------------------------------------------------------------------------------
\begin{document}
%\begin{CJK*}{UTF8}{gbsn}                          % to typeset your resume in Chinese using CJK
%-----       resume       ---------------------------------------------------------
\makecvtitle

\section{Education}
\cventry{2021--2024}{PhD thesis in Machine Learning}{Université Paris Saclay}{}{}{
	PhD thesis in Machine Learning apply to the comparison of Cytometric data.
}

\cventry{2019--2021}{Master Mathématiques et Applications}{Université Paris Saclay}{}{}{
	Master's degree in Mathematics apply to Artificial Intelligence.
}

\cventry{2016--2019}{Licence de Mathématiques}{Université Paris Saclay}{}{}{
	Bachelor's degree of Mathematics. The first two years at Evry and the last one at Orsay. 
}

\section{PhD thesis}
\cvitem{Title}{\emph{Cytometric data comparison}}
\cvitem{Supervisors}{Gilles Blanchard and Marc Glisse}
\cvitem{Abstract}{
	PhD thesis in partnership with Metafora biosystem a bio-engineering compary. The company has created a software, Metaflow, that allows automatic analysis of flow cytometry data. My work focuses on the use of machine learning models to transfer the analysis performed on one sample to a new unanalysed one. We rely on Reproducing Kernel Hilbert space to embed and store high-dimensional features in Euclidian Space. We use these representations to, firstly, estimate the proportions of each population in a new sample, and secondly to automatically name the cluster obtained by metaflow.
}

\section{Experience}
%\subsection{Vocational}
%\cventry{year--year}{Job title}{Employer}{City}{}{General description no longer than 1--2 lines.\newline{}
%Detailed achievements:
%\begin{itemize}
%\item Achievement 1
%\item Achievement 2 (with sub-achievements)
%  \begin{itemize}
%  \item Sub-achievement (a);
%  \item Sub-achievement (b), with sub-sub-achievements (don't do this!);
%    \begin{itemize}
%    \item Sub-sub-achievement i;
%    \item Sub-sub-achievement ii;
%    \item Sub-sub-achievement iii;
%    \end{itemize}
%  \item Sub-achievement (c);
%  \end{itemize}
%\item Achievement 3
%\item Achievement 4
%\end{itemize}}
%\cventry{year--year}{Job title}{Employer}{City}{}{Description line 1\newline{}Description line 2\newline{}Description line 3}
\subsection{Teaching}
\cventry{2022--2023}{Mathematics for Management}{IUT Sceaux}{L1 B.U.T GEA}{}{	
	Taught by Patrick Pamphile.
}

\subsection{Seminar}

\cventry{2022--2024}{Seminar}{Université Paris-Saclay}{Master 2}{}{
	Co-organizer of the seminar for master students in Statistics and Machine Learning at Université Paris-Saclay. 
}
\newpage
\section{Publication}

\cventry{2023}{Label Shift Quantification with Robust Guarantees via Distribution Feature Matching}{ECML/PKDD 2023, RT Track – Best paper Runner-Up amoung 192 publications}{With Gilles Blanchard and Badr-Eddine Chérief-Abdellatif}{}{
	Quantification learning deals with the task of estimating the target label distribution under label shift. There exist two main classes of quantifiers in the literature: classification-based methods vs statistical mixture modeling approaches. In this paper, we propose an efficient and scalable quantifier that belongs to the second class, and we present a unifying framework based on feature distribution matching that recovers estimators from both quantification families. In particular, we derive a general consistency theorem under label shift which improves upon the bounds that can be found in the literature, investigate the misspecified setting where the exact label shift hypothesis is challenged, and provide a detailed numerical study on simulated and real-world datasets. 
}

\section{Languages}
\cvitemwithcomment{French}{Native}{}
\cvitemwithcomment{English}{B2}{CEFRL rating}

\section{Computer skills}


\smallskip
%\cvitem{Skill matrix}{Alternatively, provide a skill matrix to show off your skills}
%% Skill matrix as an alternative to rate one's skills, computer or other. 

%% Adjusts width of skill matrix columns. 
%% Usage \setcvskillcolumns[<width>][<factor>][<exp_width>]
%% <width>, <exp_width> should be lengths smaller than \textwidth, <factor> needs to be between 0 and 1.
%% Examples:
% \setcvskillcolumns[5em][][]%    adjust first column. Same as \setcvskillcolumns[5em]
\setcvskillcolumns[][0.3][]%   adjust third (skill) column. Same as \setcvskillcolumns[][0.45]
% \setcvskillcolumns[][][\widthof{``Year''}]%     adjust fourth (years) column.
% \setcvskillcolumns[][0.45][\widthof{``Year''}]%
% \setcvskillcolumns[\widthof{``Languag''}][0.48][]
% \setcvskillcolumns[\widthof{``Languag''}]%

%% Adjusts width of legend columns. Usage \setcvskilllegendcolumns[<width>][<factor>]
%% <factor> needs to be between 0 and 1. <width> should be a length smaller than \textwidth
%% Examples:
% \setcvskilllegendcolumns[][0.45]
% \setcvskilllegendcolumns[\widthof{``Legend''}][0.45]
% \setcvskilllegendcolumns[0ex][0.46]% this is usefull for the banking style

%% Add a legend if you are using \cvskill{<1-5>} command or \cvskillentry
%% Usage \cvskilllegend[*][<post_padding>][<first_level>][<second_level>][<third_level>][<fourth_level>][<fifth_level>]{<name>}
% \cvskilllegend % insert default legend without lines
\cvskilllegend*[1em]{}% adjust post spacing
% \cvskilllegend*{Legend}%  Alternatively add a description string
%% adjust the legend entries for other languages, here German
% \cvskilllegend[0.2em][Grundkenntnisse][Grundkenntnisse und eigene Erfahrung in Projekten][Umfangreiche Erfahrung in Projekten][Vertiefte Expertenkenntnisse][Experte\,/\,Spezialist]{Legende}

%% Alternative legend style with the first three skill levels in one column
%% Usage \cvskillplainlegend[*][<post_padding>][<first_level>][<second_level>][<third_level>][<fourth_level>][<fifth_level>]{<name>}
% \setcvskilllegendcolumns[][0.6]%  works for classic, casual, banking
% \setcvskilllegendcolumns[][0.55]%  works better for oldstyle and fancy
% \cvskillplainlegend{}
% \cvskillplainlegend[0.2em][Grundkenntnisse][Grundkenntnisse und eigene Erfahrung in Projekten][Umfangreiche Erfahrung in Projekten][Vertiefte Expertenkenntnisse][Experte/Guru]{Legende}

%% Add a head of the skill matrix table with descriptions.
%% Usage \cvskillhead[<post_padding>][<Level>][<Skill>][<Years>][<Comment>]%
\cvskillhead[-0.1em]%   this inserts the standard legend in english and adjust padding
%% Adjust head of the skill matrix for other languages
% \cvskillhead[0.25em][Level][F\"ahigkeit][Jahre][Bemerkung]

%% \cvskillentry[*][<post_padding>]{<skill_cathegory>}{<0-5>}{<skill_name>}{<years_of_experience>}{<comment>}% 
%% Example usages:
\cvskillentry*{Language:}{4}{Python}{4}{Used Python and standard Machine Learning packages such as numpy, matplotlib, scikit-learn or pytorch. Creation of custom packages.}
\cvskillentry{}{2}{SQL}{}{Online course on SQLite}
\cvskillentry{}{1}{R}{1}{Used R for Machine Learning project during my education.}
\cvskillentry{}{2}{Zotero}{3}{Used Zotero during the PhD to manage my bibliography.}
\cvskillentry{}{3}{\LaTeX}{5}{}
\cvskillentry*{OS:}{3}{Linux}{4}{I only use Ubuntu for work.}
\cvskillentry{}{3}{Windows}{10+}{Use Windows at home.}

%% \cvskill{<0-5>} command
%\cvitem{\textbackslash{cvskill}:}{Skills can be visually expressed by the \textbackslash{cvskill} command, e.g. \cvskill{2}}

%\section{Interests}
%\cvitem{hobby 1}{Description}
%\cvitem{hobby 2}{Description}
%\cvitem{hobby 3}{Description}

%

\nocite{*}
\bibliographystyle{plain}
\bibliography{publications}                        % 'publications' is the name of a BibTeX file

% Publications from a BibTeX file using the multibib package
%\section{Publications}
%\nocitebook{book1,book2}
%\bibliographystylebook{plain}
%\bibliographybook{publications}                   % 'publications' is the name of a BibTeX file
%\nocitemisc{misc1,misc2,misc3}
%\bibliographystylemisc{plain}
%\bibliographymisc{publications}                   % 'publications' is the name of a BibTeX file
\end{document}


%% end of file `template.tex'.

